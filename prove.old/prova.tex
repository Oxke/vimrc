\documentclass{article}
\usepackage{layout}
\usepackage[a4paper, total={5in,9in}]{geometry}
\usepackage{mathtools}
\usepackage{amsthm}
\usepackage[framemethod=TikZ]{mdframed}
\usepackage{amsmath}
\usepackage{amssymb}
\usepackage{cancel}
\usepackage{xcolor}
\usepackage{tikz}
\usepackage{tcolorbox}
\usepackage{import}
\usepackage{pdfpages}
\usepackage{transparent}
\usepackage{xcolor}

\newcommand{\incfig}[2][1]{%
    \def\svgwidth{#1\columnwidth}
    \import{./figures/}{#2.pdf_tex}
}

\pdfsuppresswarningpagegroup=1

\newcounter{theo}[section]\setcounter{theo}{0}
\renewcommand{\thetheo}{\arabic{section}.\arabic{theo}}

\newenvironment{theorem}[1][]{
    \refstepcounter{theo}
     \ifstrempty{#1}
    {\mdfsetup{
        frametitle={
            \tikz[baseline=(current bounding box.east),outer sep=0pt]
            \node[anchor=east,rectangle,fill=blue!20,rounded corners=5pt]
            {\strut Theorem~\thetheo};}
        }
    }{\mdfsetup{
        frametitle={
            \tikz[baseline=(current bounding box.east),outer sep=0pt]
            \node[anchor=east,rectangle,fill=blue!20,rounded corners=5pt]
            {\strut Theorem~\thetheo:~#1};}
        }
    }
    \mdfsetup{
        roundcorner=10pt,
        innertopmargin=10pt,linecolor=blue!20,
        linewidth=2pt,topline=true,
        frametitleaboveskip=\dimexpr-\ht\strutbox\relax%
    }
\begin{mdframed}[]\relax}{
\end{mdframed}}

\theoremstyle{plain}
\newtheorem{lemma}[theo]{Lemma}
\newtheorem{corollary}{Corollary}[theo]
\newtheorem{proposition}{Proposition}

\theoremstyle{definition}
\newtheorem{definition}[theo]{Definition}
\newtheorem{example}{Example}

\theoremstyle{remark}
\newtheorem*{note}{Note}
\newtheorem*{remark}{Remark}

\newtcolorbox{notebox}{
  colback=gray!10,
  colframe=black,
  arc=5pt,
  boxrule=1pt,
  left=15pt,
  right=15pt,
  top=15pt,
  bottom=15pt,
}

\title{}
\author{Osea}
\date{2023-12-10}
\begin{document}

\maketitle

\begin{theorem}[THEOREMONE]
    \label{thm:1}
    Il teorema di Lebesgue
\end{theorem}

\begin{proof}
    Si può dimostrare per ovvietà,
\end{proof}

\[\frac{12}{a+b}\]

Let \(f: \mathbb{R} \to \mathbb{C}\) la funzione definita da \(x \mapsto x^2\)
Il miglior risultato che si può ottenere al momento è \(12 + 4\)


Sia \(f: \mathbb{R} \to \mathbb{R}\) definita da \(a \mapsto a^{2}_{11}  \)
possiamo calcolare il kernel della funzione \(f\) come \(\ker f = \{0\}\)
ottenendo quindi che la funzione deve essere iniettiva. Dovrà esserlo sempre
iniettiva per ogni \(a \in \mathbb{R}\) e quindi \(f\) è iniettiva.

\begin{theorem}[Iniettività \(\iff \ker = \{1\}\)]
     Sia \(f: G \to H\) un omomorfismo di gruppi da \(G\) ad \(H\). \\ Allora
     \(f\) è iniettivo se e solo se \(\ker{f}=\{1\}\).
\end{theorem}
\begin{proof}
    Sia \(f: G \to H\) un omomorfismo di gruppi da \(G\) a \(H\) allora:
    \begin{itemize}
        \item Se \(f\) è iniettava supponiamo che \(a \in \ker{f}\) allora
            \(f(a)=0\) ma dato che \(f\) è un omomorfismo allora \(f(0)=0\)
            quindi per l'iniettività \(a=0\) e quindi \(\ker{f}=\{1\}\)
        \item Se \(\ker{f}=\{1\}\) allora \(f\) è iniettiva. Infatti se
            \(f(a)=f(b)\) allora \(f(a-b)=0\) e quindi \(a-b=0\) e quindi
            \(a=b\)
    \end{itemize}
\end{proof}

\begin{figure}[ht]
    \centering
    \incfig{test-picture}
    \caption{test picture}
    \label{fig:test-picture}
\end{figure}

\section{Radice di 2 è irrazionale}
\begin{theorem}
    \label{thm:2}
    \(\sqrt{2}\) è irrazionale
\end{theorem}
\begin{proof}
    Supponiamo per assurdo che \(\sqrt{2}\) sia razionale, allora
    \(\sqrt{2}=\frac{a}{b}\) con \(a,b \in \mathbb{Z}\) e \(b \neq 0\). \\
    Possiamo supporre che \(\frac{a}{b}\) sia ridotta ai minimi termini,
    cioè che \(a\) e \(b\) siano coprimi. \\
    Allora \(\sqrt{2}=\frac{a}{b}\) e quindi \(2=\frac{a^2}{b^2}\) e quindi
    \(a^2=2b^2\). \\
    Quindi \(a^2\) è pari e quindi \(a\) è pari. \\
    Allora \(a=2k\) con \(k \in \mathbb{Z}\) e quindi \(2b^2=4k^2\) e quindi
    \(b^2=2k^2\) e quindi \(b^2\) è pari e quindi \(b\) è pari. \\
    Ma questo è assurdo dato che abbiamo supposto che \(a\) e \(b\) fossero
    coprimi. \\
    Quindi \(\sqrt{2}\) è irrazionale.
\end{proof}

\begin{figure}[ht]
    \centering
    \incfig[.4]{faccia}
    \incfig[.4]{funzione-convessa}
    \caption{faccia}
    \label{fig:faccia}
\end{figure}

let \(f: \mathbb{R} \to \mathbb{R}\)

\[
    \coprod_{i \in  I} 
\]

\[
    ds^2 = -c\,dt^2 + dx^2 + dy^2 + dz^2
\]

\begin{align*}
    f: \mathbb{R} &\longrightarrow \mathbb{C} \\
    x &\longmapsto f(x) = \frac{1}{2\pi i} \int_{\gamma} \frac{1}{z-x} dz
\end{align*}

so we know that \(e^{2x^2} = A \subset B \text{24}\)

quindi otteniamo alla fine che \(A^{\mathcal{I}} = \varnothing \subset\)
everything so also \(B^{\mathcal{I}}\)

\textbf{Il migliore} risultato che fosse mai stato trovato
so:
\begin{align*}
    a &= b + c \\
    \iff a - b &= c \\
    \text{da cui otteniamo che}
    \implies a - b &= c \\
    \text{e a sua volta} a - c &= b \impliedby 
\end{align*}

\[
    \frac{\left( \frac{1}{2} \right)}{2}
\]
\end{document}
